\documentclass[a4paper,10pt]{article}
\usepackage[utf8]{inputenc}
\usepackage{color}
%opening
\title{}
\author{}

\begin{document}

\maketitle

\begin{abstract}

\end{abstract}

\section{}

\subsection{ARPACK eigensolver and ARPACK-CG linear solver}
The Arnoldi Package (ARPACK) \cite{arpack-page} is a well known FORTRAN library for solving large scale eigenvalue problems. It can solve standard as well as generalized
eigenvalue problems. An interface for using ARPACK inside tmLQCD is added under the solver subdirectory and is called {\tt eigenvalues\_arpack}. This interface was used
for a new solver called {\tt arpack\_cg} in which we use eignvectors computed with ARPACK to deflate CG. The solver interface has some common features with eigCG. A sample
input is also provided. A more detailed description is also given inside the corresponding header and source files. 

In order to configure tmLQCD with ARPACK, it is needed that ARPACK and (P)ARPACK (the MPI parallel version) are installed. Assuming that these libraries are installed 
in a directory {\tt arpack\_dir} under the names {\tt libarpack.a} and {\tt libparpack.a}, then in the simplest case one needs to add these libraries to {\tt LIBS} 
and {\tt arpack\_dir} to {\tt LDFLAGS}. It might be needed to also add {\tt -lgfortran} to {\tt LIBS}. That should be sufficent to compile tmLQCD with arpack.



\begin{thebibliography}{999}
\bibitem{arpack-page}
 http://www.caam.rice.edu/software/ARPACK/
\end{thebibliography}




\end{document}